\section{Conclusion}
\label{section:discussion}

This work introduces the formal concept of similarity inclusion dependencies (sINDs), extending traditional inclusion dependencies with a similarity measure.
First, we identified use cases for sINDs, which include discovering foreign-key candidates and join partners.
Second, we formalized an sIND definition that extends traditional INDs with an arbitrary similarity measure.
Third, we presented \sawfish, the first efficient approach to automatically discover sINDs from data.
It finds all unary sINDs based on the edit-distance and the Jaccard similarity measure.
\sawfish combines approaches of traditional IND discovery and string similarity joins with a novel sliding-window approach and lazy candidate validation.
Fourth, we evaluated \sawfish, showing that it scales well in the number of rows, and in the number of columns.
Compared to a baseline implementation, we outperformed it by a factor of up to~6.5.
Finally, we examined the sINDs discovered by \sawfish and observed real-world examples indicating joinability.

While sIND discovery is a harder problem than traditional IND discovery, the runtime could be further improved by multithreading or distributing the process.
As \algorithmName{Sindy}~\cite{dursch2019eval} demonstrated, distribution can significantly improve the performance.
However, we observed that a single column or even a single sIND candidate can dominate the runtime.
Therefore, it is not trivial to scale \sawfish's approach to multiple threads or nodes.
Further future work shall extend \sawfish to discover n-ary sINDs.
